The elementary operations are:
\begin{itemize}
	\item read of a single object $x$ ($r(x)$);
	\item write of a single data object $x$ ($w(x)$).
\end{itemize}
They may require reading from disk or writing to disk an entire page.

The \emph{scheduler} is a block of the concurrency control manager and it's in charge of deciding if and when read/write requests can be satisfied.
The \emph{transaction} is a sequence of read and write operations characterized by the same transaction id (\emph{TID}).
The \emph{schedule} is a sequence of read/write operations presented by concurrent transactions; operations in
the schedule appear in the arrival order of requests.
The concurrency control accepts or rejects schedules to avoid anomalies (lost update, dirty read, inconsistent read, ghost update).
The scheduler has to accept or reject operation execution withoutknowing the outcome of the transactions (abort or commit).
The \emph{serial schedule} is a schedule in which the actions of each transaction appear in sequence, without interleaved actions belonging to different transactions.
An arbitrary schedule $S_i$ is correct when it yields the same result as an arbitrary serial schedule $S_j$ of the same transaction and in this case it's possible to say that $S_i$ is \emph{serializable} (it's equivalent to an arbitrary serial schedule of the same transactions).
It's possible to define different equivalence classes between two schedules and they are:
\begin{itemize}
	\item view equivalence (VSR);
	\item conflict equivalence (CSR);
	\item 2-phase locking (2PL);
	\item timestamp equivalence.
\end{itemize}
Each equivalence class detects a set of acceptable schedules and they are characterized by a different complecity in detecting equivalence.

\subsubsection{View Equivalence (VSR)}
Definitions:
\begin{description}
	\item[read-from] $r_i(x)$ reads-from $w_j(x)$ when $w_j$ precedes $r_i$ and $i \neq j$ and when there's no other $w_k$ between them;
	\item[final write] $w_i(x)$ is a final write if it's the last write of $x$ appearing in the schedule.
\end{description}
Two schedules are \emph{view equivalent} if they have the same read-from set and the same final write set.
A schedule is \emph{view serializable} if it's view equivalent to an arbitrary serial schedule of the same transactions.
Detecting view equivalence to a given schedule has linear complexity.
Detecting view equivalence to an arbitrary serial schedule is \emph{NP-complete} problem (checking for \underline{all} possible permutations of an arbitrary schedule).

\subsubsection{Conflict Equivalence (CSR)}
Action $A_i$ is in conflict with action $A_j$ ($i \neq j$) if both actions operate on the same object and at least one of them is a write (\emph{read-write} conflict (RW) or \emph{write-read} conflict (WR), \emph{write-write} conflict
(WW)).
Two schedules are \emph{conflict equivalent} if they have the same conflict set and each conflict pair is in the same order in both schedules.
A schedule is \emph{conflict serializable} if it's equivalent to an arbitrary serial schedule of the same transactions.
Conflict graph: it's used to detect conflict serializability.
It's represented in this way: a node for each transaction, an edge $T_i \to T_j$ if there exists at least a conflict between an action $A_i$ in $T_i$ and $A_j$ in $T_j$ and $A_i$ precedes $A_j$.
If this conflict graph is acyclic then the schedule is \emph{conflict serializable} (CSR).
Checking Conflict Graph is a linear operation.

\subsubsection{2-Phase locking (2PL)}
A \emph{lock} is a block on a resource which may prevent access to others.
Lock operations can be: \emph{lock} (read lock or write lock) or \emph{unlock}.
Each read (write) operation is preceded by a request of read (write) lock and it's followed by a request of unlock.
The difference between \emph{R-Lock} and \emph{W-Lock} is that in the first case the lock is shared among different transactions, while the \emph{W-Lock} is exclusive and only one transaction per time can obtain the \emph{W-Lock} on a specific resource.
There's another situation that is the \emph{lock escalation}: it's a particular situation in which a transaction requests an \emph{R-Lock} followed by \emph{W-Lock} on the same specific resource.
With this method, the scheduler becomes a \emph{lock manager}.
It receives transaction requests and grants locks based on locks already granted to other transactions:
\begin{itemize}
	\item when the lock request is granted then the corresponding resource is acquired by the requesting transaction and when the transaction performs the unlock, the resource becomes again available;
	\item when the lock is not granted the requesting transaction is put in a waiting state and this state terminates when the resource is unlocked and become available.
\end{itemize}
The lock manager exploits the information in the lock table to decide if a given lock can be granted to a transaction and the conflict table to manage lock conflicts (conflict table summarizes the rules to grant a lock).

\begin{table}[h]
	\centering
	\begin{tabular}{c|l|l|l}
		\multicolumn{1}{c|}{\multirow{2}{*}{Request}} & \multicolumn{3}{c}{Resource State} \\ \cline{2-4}
		& Free & R-Locked & W-Locked \\ \hline
		R-Lock & OK/R-Locked & OK/R-Locked & Nope/W-Locked \\ \hline
		W-Lock & OK/W-Locked & \parbox[l]{3cm}{Nope/R-Locked\\or OK/W-Locked\\(for lock escalation)} & Nope/W-Locked \\ \hline
		Unlock & ERROR & OK/R-Locked of Free & OK/Free \\
	\end{tabular}
	\caption{Conflict Table}
\end{table}
Read locks are shared and other transactions may lock the same resource; for this reason a counter is used to count the number of transactions currently holding the R-Lock (the resource is free when the counter is zero).

The two phases are:
\begin{description}
	\item[growing phase] this is the initial phase of each transaction, in which all needed locks are acquired;
	\item[shrinking phase] this is the last phase of each transaction, in which all locks are released.
\end{description}
It's important to notice that if a transactions starts with the shrinking phase, it cannot acquire other resources; in other words: a transaction cannot acquire a now lock after having released any lock.

A variant of 2-Phase Locking is the \emph{Strict 2-Phase Locking} in which a transaction locks may be released only at the end of the transaction (after commit/rollback).
The process is articulated in this way: a transaction requests a resource $x$, if the request can be satisfied, the lock manager modify the state of resource $x$ in its internal tables and it returns control to the requesting transaction; if the request cannot be satisfied immediately, the requesting transaction in inserted in a waiting queue and suspended (wait) and only when the resource become available, the first transaction in the waiting queue is resumed and is granted the lock on the resource.
When a timeout of a generic transaction in the waiting queue expires, the lock manager extracts the waiting transaction from the queue, resumes it and returns a not ok error code, the requesting transaction can perform a rollback and request again the same lock after some time.
The timeout for each transaction is useful also to avoid deadlocks.